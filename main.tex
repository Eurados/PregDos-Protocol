\documentclass[11pt]{article}
\usepackage[a4paper,margin=1in]{geometry}
\usepackage{hyperref}
\usepackage{longtable}
\usepackage{graphicx}
\usepackage{fancyhdr}
\usepackage{listings}
\usepackage{titlesec}
\usepackage{enumitem}
\usepackage[sort&compress,numbers,square]{natbib}

\pagestyle{fancy}
\fancyhf{}
\rhead{PregDose Protocol}
\lhead{SONORA Project}
\rfoot{\thepage}

\titleformat{\section}{\normalfont\Large\bfseries}{\thesection}{1em}{}
\titleformat{\subsection}{\normalfont\large\bfseries}{\thesubsection}{1em}{}

\title{\textbf{PregDose: Protocol and Software Structure}\\Neutron Dose Estimation in Proton Therapy}
\author{Developed as part of the SONORA Project \\ Grant Agreement No. 101061037}
\date{\today}

\begin{document}
\maketitle

\section{Purpose}

\textbf{PregDose} is a clinical tool designed to estimate neutron equivalent dose to the fetus during proton therapy. It converts DICOM-based treatment planning data into input scripts for TOPAS Monte Carlo simulations~\cite{perl_topas_2012}, supporting accurate field-level neutron dose estimation.

\section{Overview}

PregDose:
\begin{itemize}[topsep=0pt,itemsep=1pt]
    \item Parses a full DICOM treatment study including CT, RTSTRUCT, RTPLAN, and RTDOSE.
    \item Uses external beam model and SPR-to-material tables.
    \item Outputs TOPAS-compatible input files for dose simulation.
    \item Supports field-specific geometry generation and dose verification.
\end{itemize}

\section{System Requirements}
\begin{itemize}
    \item Python $\geq 3.8$
    \item TOPAS (installed and configured)
    \item Python packages: \texttt{pydicom}, etc. (see \texttt{setup.py})
\end{itemize}

\section{Installation}

\begin{lstlisting}[language=bash]
git clone https://github.com/your-org/pregdose.git
cd pregdose
python3 -m venv .venv
source .venv/bin/activate
pip install -e .
\end{lstlisting}

When prompted, select ``yes'' to install all options.

\section{Input Requirements}

The input directory (\texttt{study\_dir}) must include:

\begin{longtable}{|l|l|l|}
\hline
\textbf{Type} & \textbf{Description} & \textbf{Format} \\
\hline
CT Series & DICOM image data & CT\texttt{*.dcm} \\
RS File & RTSTRUCT (structure set) & RS\texttt{*.dcm} \\
RN File & RTPLAN (treatment plan) & RN\texttt{*.dcm} \\
RD File & RTDOSE (TPS dose) & RD\texttt{*.dcm} \\
\hline
\end{longtable}

Additional required files:

\begin{longtable}{|l|l|l|}
\hline
\textbf{Parameter} & \textbf{Description} & \textbf{Example} \\
\hline
\texttt{-b} / \texttt{--beam-model} & Beam model CSV & \texttt{DCPT\_beam\_model\_\_v2.csv} \\
\texttt{-s} / \texttt{--spr-to-material} & SPR to material table & \texttt{SPRtoMaterial\_\_Brain.txt} \\
\texttt{-p} / \texttt{--beam-model-position} & Beam model distance to isocenter (mm) & \texttt{500.0} \\
\hline
\end{longtable}

\section{Usage Example}

\begin{lstlisting}[language=bash]
PYTHONPATH=. python3 pregdos/main.py \
  -v \
  -b=res/beam_models/DCPT_beam_model__v2.csv \
  -p 500.0 \
  -s=res/spr_tables/SPRtoMaterial__Brain.txt \
  res/test_studies/DCPT_headphantom/
\end{lstlisting}

Generates:
\begin{itemize}
    \item \texttt{topas\_field1.txt}
    \item \texttt{topas\_field2.txt}
    \item \texttt{topas\_field3.txt}
\end{itemize}

\section{Command-Line Options}

\begin{lstlisting}[language=bash]
usage: main.py [-h] [-b BM] [-s SPR_TO_MATERIAL_PATH] [-p BEAM_MODEL_POSITION]
               [-f FIELD_NR] [-N NSTAT] [-v] [-V]
               study_dir [output_base_path]
\end{lstlisting}

\begin{longtable}{|l|l|p{8cm}|}
\hline
\textbf{Option} & \textbf{Required} & \textbf{Description} \\
\hline
\texttt{study\_dir} & Yes & Path to input folder (CT, RS, RN, RD) \\
\texttt{output\_base\_path} & No & Output base name (default: \texttt{topas.txt}) \\
\texttt{-b} / \texttt{--beam-model} & Yes & Beam model CSV file \\
\texttt{-s} / \texttt{--spr-to-material} & Yes & SPR-to-material mapping file \\
\texttt{-p} / \texttt{--beam-model-position} & No & Beam model position (mm) \\
\texttt{-f} / \texttt{--field} & No & Export only a single field \\
\texttt{-N} / \texttt{--nstat} & No & Number of primary protons \\
\texttt{-v} & No & Verbose output \\
\texttt{-V} & No & Show version and exit \\
\hline
\end{longtable}

\section{Output}

PregDose produces one TOPAS input file per treatment field, including:
\begin{itemize}
    \item Voxelized CT geometry
    \item Structure definitions (e.g. fetus ROI)
    \item Field and beam configuration
\end{itemize}

\section{Internal Module Structure}

\begin{longtable}{|l|p{10cm}|}
\hline
\textbf{Module} & \textbf{Functionality} \\
\hline
\texttt{pregdos/main.py} & CLI parsing and main execution logic \\
\texttt{import\_rtstruct.py} & Handles RTSTRUCT parsing, extracts ROIs \\
\texttt{import\_rtplan.py} & Parses RTPLAN and beam geometry \\
\texttt{export\_study\_topas.py} & Converts study data to TOPAS geometry files \\
\texttt{utils/} & Shared utilities and helpers \\
\hline
\end{longtable}

\section{Dose Verification}

PregDose reads RTDOSE (TPS dose grids) to allow for Monte Carlo vs TPS dose comparison. This supports:
\begin{itemize}
    \item QA for simulation accuracy
    \item Clinical research into fetal dose metrics
\end{itemize}

\section{Fetus ROI Scoring}

Ensure the fetus ROI is properly defined in the RTSTRUCT file. PregDose will:
\begin{itemize}
    \item Isolate the ROI using contour data
    \item Define a scoring volume for neutron dose estimation
\end{itemize}

\section{Licensing and Acknowledgements}

This project is part of the \textbf{SONORA} initiative, funded by:

\begin{quote}
\textit{The European Union’s EURATOM research and innovation programme under Grant Agreement No. 101061037 (PIANOFORTE – European Partnership for Radiation Protection Research).}
\end{quote}

\section{Planned Features}

\begin{itemize}
    \item GPU-accelerated simulation integration
    \item Automated fetus detection and scoring
    \item 3D visualization of dose volumes
    \item GUI-based front-end for clinical researchers
\end{itemize}

\bibliographystyle{ieeetr}
\bibliography{bibliography}

\end{document}
