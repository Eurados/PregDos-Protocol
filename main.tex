\documentclass[11pt]{article}
\usepackage[a4paper,margin=1in]{geometry}
\usepackage{hyperref}
\usepackage{longtable}
\usepackage{graphicx}
\usepackage{fancyhdr}
\usepackage{listings}
\usepackage{titlesec}
\usepackage{enumitem}
\usepackage[sort&compress,numbers,square]{natbib}

\pagestyle{fancy}
\fancyhf{}
\rhead{PregDose Protocol}
\lhead{SONORA Project}
\rfoot{\thepage}

\titleformat{\section}{\normalfont\Large\bfseries}{\thesection}{1em}{}
\titleformat{\subsection}{\normalfont\large\bfseries}{\thesubsection}{1em}{}

\title{\textbf{PregDose: Protocol and Software Structure}\\Neutron Dose Estimation in Proton Therapy}
\author{Developed as part of the SONORA Project \\ Grant Agreement No. 101061037}
\date{\today}

\begin{document}
\maketitle

\section{Purpose}

\textbf{PregDose} is a clinical tool intended for recalculating proton therapy plans for pregnant patients and estimate neutron equivalent dose to the fetus. It converts DICOM-based treatment planning data into input-files for the TOPAS Monte Carlo simulation tool~\cite{perl_topas_2012}, with the intend to provide clinicians quantitative radiation safety parameters to aid in the treatment planning process. 

\section{Installation}
The software tool can be installed from the GitHub repository directly \url{https://github.com/Eurados/pregdos} or as a Docker container at XXXX. When installing it from GitHub do:
\begin{lstlisting}[language=bash]
git clone https://github.com/Eurados/pregdos
cd pregdose
python3 -m venv .venv
source .venv/bin/activate
pip install -e .
\end{lstlisting}

When prompted, select ``yes'' to install all options.


\subsection{System Requirements}
\begin{itemize}
    \item Python $\geq 3.8$
    \item TOPAS (installed and configured)
    \item Python packages: \texttt{pydicom}, etc. (see \texttt{setup.py})
\end{itemize}

\section{Input Requirements}

The input directory (\texttt{study\_dir}) must include the following set of files:

\begin{longtable}{|l|l|}
\hline
\textbf{Description} & \textbf{Format} \\
\hline
DICOM image data & CT\texttt{*.dcm} \\
RTSTRUCT (structure set) & RS\texttt{*.dcm} \\
RTPLAN (treatment plan) & RN\texttt{*.dcm} \\
RTDOSE (TPS dose) & RD\texttt{*.dcm} \\
\hline
\end{longtable}

Here the RTDOSE file can be omitted, but is recommended as the simulations also generate dose files for verifying with the plans from the TPS. The RS structure set, needs to include a structure with the name "fetus" such that relevant neutron specific quantities can be scored correctly. 

Besides the required files it is recommended to include additional machine specific information to ensure the most reliable results. These include a beam model, CT or SPR to material conversion and the distance in which the beam model is defined from. Below you find a table on how to includes this in the program and examples of files/values with can be found in the GitHub repository. 

\begin{longtable}{|l|l|l|}
\hline
\textbf{Flag} & \textbf{Description} & \textbf{Example} \\
\hline
\texttt{-b} / \texttt{--beam-model} & Beam model CSV & \texttt{DCPT\_beam\_model\_\_v2.csv} \\
\texttt{-s} / \texttt{--spr-to-material} & SPR to material table & \texttt{SPRtoMaterial\_\_Brain.txt} \\
\texttt{-p} / \texttt{--beam-model-position} & Beam model distance to isocenter (mm) & \texttt{500.0} \\
\hline
\end{longtable}

\section{Usage Example}

If you have installed PregDos through the GitHub page, you can try the example below to get experince with the program. Note that all necessary files are included in the GitHub repository.

\begin{lstlisting}[language=bash]
PYTHONPATH=. python3 pregdos/main.py \
  -v \
  -b=res/beam_models/DCPT_beam_model__v2.csv \
  -p 500.0 \
  -s=res/spr_tables/SPRtoMaterial__Brain.txt \
  res/test_studies/DCPT_headphantom/
\end{lstlisting}

This the example plan is a three field plan, the program generates and runs three simulations, one for each field.
\begin{itemize}
    \item \texttt{topas\_field1.txt}
    \item \texttt{topas\_field2.txt}
    \item \texttt{topas\_field3.txt}
\end{itemize}

These files can be modified by the user prior to running, if needed.  

\section{Output}

PregDose produces one TOPAS input file per treatment field, including:
\begin{itemize}
    \item Voxelized CT geometry
    \item Structure definitions (e.g. fetus ROI)
    \item Field and beam configuration
\end{itemize}

PregDose reads RTDOSE (TPS dose grids) to allow for Monte Carlo vs TPS dose comparison. This supports:
\begin{itemize}
    \item QA for simulation accuracy
    \item Clinical research into fetal dose metrics
\end{itemize}

\section{Command-Line Options}

\begin{lstlisting}[language=bash]
usage: main.py [-h] [-b BM] [-s SPR_TO_MATERIAL_PATH] [-p BEAM_MODEL_POSITION]
               [-f FIELD_NR] [-N NSTAT] [-v] [-V]
               study_dir [output_base_path]
\end{lstlisting}

\begin{longtable}{|l|l|p{8cm}|}
\hline
\textbf{Option} & \textbf{Required} & \textbf{Description} \\
\hline
\texttt{study\_dir} & Yes & Path to input folder (CT, RS, RN, RD) \\
\texttt{output\_base\_path} & No & Output base name (default: \texttt{topas.txt}) \\
\texttt{-b} / \texttt{--beam-model} & Yes & Beam model CSV file \\
\texttt{-s} / \texttt{--spr-to-material} & Yes & SPR-to-material mapping file \\
\texttt{-p} / \texttt{--beam-model-position} & No & Beam model position (mm) \\
\texttt{-f} / \texttt{--field} & No & Export only a single field \\
\texttt{-N} / \texttt{--nstat} & No & Number of primary protons \\
\texttt{-v} & No & Verbose output \\
\texttt{-V} & No & Show version and exit \\
\hline
\end{longtable}


\section{Internal Module Structure}

\begin{longtable}{|l|p{10cm}|}
\hline
\textbf{Module} & \textbf{Functionality} \\
\hline
\texttt{pregdos/main.py} & CLI parsing and main execution logic \\
\texttt{import\_rtstruct.py} & Handles RTSTRUCT parsing, extracts ROIs \\
\texttt{import\_rtplan.py} & Parses RTPLAN and beam geometry \\
\texttt{export\_study\_topas.py} & Converts study data to TOPAS geometry files \\
\texttt{utils/} & Shared utilities and helpers \\
\hline
\end{longtable}



\section{Licensing and Acknowledgements}

This project is part of the \textbf{SONORA} initiative, funded by:

\begin{quote}
\textit{The European Union’s EURATOM research and innovation programme under Grant Agreement No. 101061037 (PIANOFORTE – European Partnership for Radiation Protection Research).}
\end{quote}

\section{Planned Features}

\begin{itemize}
    \item GPU-accelerated simulation integration
    \item Automated fetus detection and scoring
    \item 3D visualization of dose volumes
    \item GUI-based front-end for clinical researchers
\end{itemize}

\bibliographystyle{ieeetr}
\bibliography{bibliography}

\end{document}
